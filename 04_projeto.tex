\section{Projeto de pesquisa}

\begin{frame}{Projeto de pesquisa}
	\begin{alertblock}{Projeto de pesquisa }
	\end{alertblock}
\end{frame}



\begin{frame} {Projeto - Hipóteses}

Este trabalho considera as seguintes hipóteses:

\begin{block}{H1:} O uso de modelos independentes de idioma do tipo de distorção textual permite filtrar aspectos específicos do texto, e a combinação de diversos tipos de distorção pode aumentar o desempenho de sistemas de AA.
\end{block}

\begin{block}{H2:} O uso de modelos dependentes de idioma do tipo {\it part-of-speech} extraídos por anotadores baseados em aprendizado profundo pode aumentar o desempenho de sistemas de AA.
\end{block}


\begin{block}{H3:} O uso de modelos dependentes de idioma do tipo representação distribuída ({\it embeddings}) pode aumentar o desempenho de sistemas de AA.
\end{block}

\end{frame}

\begin{frame}{Projeto - Objetivo}
\begin{block}{Objetivo Geral}
O objetivo geral deste trabalho é enriquecer modelos de atribuição autoral de texto digitais com conjunto fechado de autores utilizando conhecimentos dependentes e independentes de idioma combinados com técnicas de aprendizados de máquina, de modo a obter resultados superiores ao estabelecido em trabalhos anteriores.
\end{block}
\end{frame}


\begin{frame}{Projeto - Conjunto de dados e Avaliação}

\begin{block}{Avaliação das hipóteses}
	\begin{itemize}
		\item Comparação com {\it baselines} pertinentes, como o modelo apresentado em \citeonline{custodioParaboni2018}.
		\item Serão utilizadas as medidas tradicionais de AM, como {\it medida F}, acurácia, auroc e outros.
		\item Espera-se que o resultado médio seja superior ao dos modelos de {\it baseline}.
	\end{itemize}
\end{block}

\setlength{\tabcolsep}{4pt}\selectFont
\begin{table}[!htbp]
	\centering
	\caption{Córpus para avaliação dos métodos de AA}
	\begin{tabular}{l|rll}
		\toprule
		Córpus       & No. Autores & Idioma             & Domínio/Gênero \\ \midrule
		PAN-CLEF2014 &           - & EN, ES, DU, GR     & NV, AR, RV, ES \\ \hline
		PAN-CLEF2018 &          20 & EN, ES, FR, IT, PL & NV             \\ \hline
		RCV1         &          50 & EN                 & AR             \\ \hline
		Nus-SMS      &         116 & EN                 & SMS            \\ \hline
		b5-post      &       1.019 & PT-Br              & Facebook       \\ \hline
		BlogSet-BR   &       4.331 & PT-Br              & AR             \\ \bottomrule
	\end{tabular}
	\label{tab.results}
\end{table}

\end{frame}

\begin{frame}{Projeto - Escopo e limitações}
	Este projeto de pesquisa se limita
	\begin{itemize}
		\item ao estudo das técnicas de distorção textual
		\item ao estudo das técnicas de anotações linguísticas
		\item ao estudo das técnicas de representação distribuída
		\item e utilizará métodos de aprendizado de máquina.
		\item aos idiomas considerados primordialmente são inglês e português brasileiro.
	\end{itemize}

	Não serão considerados
	\begin{itemize}
		\item  modelos computacionais baseados em grafos, redes complexas e modelos de compressão.
	\end{itemize}
\end{frame}


\begin{frame}{Projeto - Atividades}

\begin{enumerate}
	\item {\bf Revisão bibliográfica} Concluído.
	
	\item {\bf Participação na PAN-CLEF 2018} Concluído.
	
	\item {\bf Preparação dos dados} Concluído.
	
	\item {\bf Modelos independentes de idioma} Estudo dos tipo de distorção textual, construção dos modelos computacionais e refinamentos específicos.
	
	\item {\bf Modelos baseados em anotações} Estudo dos pacotes para anotações POS, como NLTK\footnote{Disponível em  \url{https://www.nltk.org/}} e Spacy, construção dos modelos computacionais e refinamentos específicos.
	
	\item {\bf Modelos baseados em {\it embeddings}} Preparação de bases de dados de {\it embeddings}, estudo de {\it embeddings} específicos para AA, construção de modelos computacionais e refinamentos.
	
	\item {\bf Refinamentos} 
	
	\item {\bf Avaliação}
	
	\item {\bf Redação da dissertação} 
	
	\item {\bf Divulgação} 
\end{enumerate}
\end{frame}

\begin{frame}{Projeto - Cronograma}


\setlength{\tabcolsep}{4pt}\selectFont
\begin{table}[]
	\centering
	\caption{Cronograma}
		\begin{tabular}{|l|l|l|l|l|l|l|l|l|l|l|l|l|l|}
			\toprule
			                                         & \multicolumn{7}{c|}{2018}      & \multicolumn{6}{c|}{2019} \\ \midrule
			Atividades                               & 1-6 & 7 & 8 & 9 & 10 & 11 & 12 & 1 & 2 & 3 & 4 & 5 & 6     \\ \hline
			01. Revisão bibliográfica                & x   & x & x & x & x  &    &    &   &   &   &   &   &       \\ \hline
			02. Participação PAN-CLEF2018            & x   &   &   &   &    &    &    &   &   &   &   &   &       \\ \hline
			03. Preparação dos dados                 & x   & x &   &   &    & x  &    &   &   &   &   &   &       \\ \hline
			04. Modelos independentes de idioma      &     &   &   &   &    & x  & x  & x &   &   &   &   &       \\ \hline
			05. Modelos baseados em anotações        &     & x &   &   &    & x  & x  & x &   &   &   &   &       \\ \hline
			06. Modelos baseados em {\it embeddings} &     & x &   &   &    &    &    & x & x &   &   &   &       \\ \hline
			07. Refinamentos                         &     &   &   &   &    &    &    & x & x & x &   &   &       \\ \hline
			08. Avaliação final                      &     &   &   &   &    &    &    &   &   & x &   &   &       \\ \hline
			09. Redação da dissertação               &     &   &   &   &    &    &    &   &   & x & x & x &       \\ \hline
			10. Divulgação                           &     &   &   &   &    &    &    &   &   &   &   & x & x     \\ \bottomrule
		\end{tabular}
	\end{table}
\end{frame}


