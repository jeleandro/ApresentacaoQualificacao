\section{Trabalhos relacionados}

\begin{frame}{Trabalhos relacionados}

Foram estudados os trabalhos:
\begin{itemize}
	\item relacionados ao histórico da competição PAN-CLEF.
	\item trabalhos recentes encontrados no Scopus, IEEE e ACL Anthology
	\item que usaram n-gramas, embeddings e variações.
	\item que usaram técnicas de similaridade/distância/vizinhos mais próximos.
	\item que usaram métodos de distância específicos para AA.
	\item que usaram redes neurais.
	\item que analisaram línguas europeias.
\end{itemize}

\end{frame}

\begin{frame}{Trabalhos selecionados}
\setlength{\tabcolsep}{5pt}\selectFont
\begin{table}[]
	\caption{Trabalhos selecionados}
	\begin{tabular}{lllll}
		\toprule
		{ Estudo}                   & {  Idioma}     & { Tarefa} & {  Conhecimento}      & {  Método}           \\ \toprule
		\citeonline{aa-Sapkota2015} & EN             & A         & {\it C}               & SVM                  \\ \hline
		\citeonline{aa-distortion}  & EN             & A,V       & {\it C, W}            & SVM                  \\ \midrule
		\citeonline{Schwartz2013}   & EN             & A         & {\it C, W}            & SVM                  \\ \hline
		\citeonline{aa-rocha-2017}  & EN             & A,V       & {\it C, W, P}         & {\it SVM}, RF e SCAP \\ \midrule
		\citeonline{AA_delta2017}   & EN             & C         & {\it W}               & Clusterização        \\ \hline
		\citeonline{Varela2016}     & PT-BR          & A,V       & {\it P}               & SVM                  \\ \hline
		\citeonline{posadas2017}    & EN             & A         & D2V de {\it W}        & Softmax e SVM        \\ \hline
		\citeonline{RhodesCS224D}   & EN             & A         & W2V                   & CNN-Softmax          \\ \hline
		\citeonline{Shrestha2017}   & EN             & A         & {\it C One-hot}       & Softmax              \\ \hline
		\citeonline{Bagnall2016}    & PAN2015 & C         & {\it C One-hot}       & RNN-Softmax          \\ \bottomrule
	\end{tabular}
	\label{tab:revisao_sumarizacao_geral}
	\SourcePadrao
\end{table}
\end{frame}

\begin{frame}{Trabalhos relacionados - Considerações}

\begin{itemize}
	\item Os métodos de similaridade representam uma ferramenta importante para AA.
	\item As {\bf redes de convolução} apresentaram resultados equivalentes aos {\it baselines}, entretanto, apresentaram custo computacional maior.
	\item As {\bf redes recorrentes} apresentaram desempenho superiores ao {\it baseline}, no entanto, apresentaram custo computacional elevado e precisou de dados adicionais.
	\item {\bf Embeddings pré-treinados} obtiveram desempenho equivalente ao {\it baseline}.
	\item As técnicas de aprendizado profundo possuem aplicações limitadas na AA porque nem sempre é possível ter um volume de dados expressivo de forma a garantir a estabilidade dos métodos.
	\item Os modelos baseados em caracteres representaram desempenho consistentes.
	\item Os modelos que usam distorção se demonstraram promissores.
	
\end{itemize}

\end{frame}


