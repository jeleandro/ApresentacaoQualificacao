\section{Introdução}

\begin{frame}{Introdução}
	\begin{alertblock}{Contexto}
		\begin{itemize}\itemsep9pt
		\item A atribuição autoral de textos digitais (AA) (do inglês, {\it Authorship Attribution})  visa identificar quem é o autor de um determinado texto a partir de um conjunto de autores possíveis \cite{Potthast2017}.
		
		\item A premissa principal da AA é que o autor deixa rastros de seu estilo, sendo que esses rastos podem ser a preferência por certas palavras, o tamanho do vocabulário, a utilização de pontuação e a repetição de certos elementos gramaticais.
		
		\item A quantificação do estilo de escrita, ou estilometria, compreende um vasto conjunto de medidas e técnicas que buscam extrair uma ``biometria'' textual \cite{Neal2017}.
		
		\end{itemize}
	
		
	\end{alertblock}

\end{frame}

\begin{frame}{Aplicações da Atribuição Autoral}

\begin{columns}
	\begin{column}{0.65\textwidth}
		\begin{block}{Aplicações da AA}
			Sua aplicação pode ajudar: 
			\begin{itemize}\itemsep9pt
				\item  em casos de escândalos de corrupção, como no caso Enron
				 \cite{corpusEnron,Chen2011}.
				\item na identificação de abusos na utilização da internet \cite{Gillam2012quite}.
				\item na detecção de notícias falsas \cite{Peng2016}.
				\item na detecção de casos onde uma pessoa tenta se passar por outra \cite{Koppel2018_pseud}.
				\item na atribuição autoral de código-fonte \cite{Alsulami2017}
				\item na detecção de pseudônimos \cite{Juola15}
				
			\end{itemize}
		\end{block}
	\end{column}
	\begin{column}{0.3\textwidth}
		\begin{block}{Áreas de interesse}
			\begin{itemize}
				\item Humanidades Digitais
				\item Análise Forense
				\item Linguística computacional
			\end{itemize}
		\end{block}
	\end{column}
\end{columns}

\end{frame}


\begin{frame}{Métodos para atribuição autoral}

Os métodos computacionais para atribuição autoral utilizam: 
\begin{itemize}
  \item Análise estatística multivariada \cite{Savoy2016,AA_delta2017}.
  \item Métodos baseados em vizinho mais próximo \cite{Kocher2017Verificacao,Koppel2018_pseud,Varela2016}.
  \item Aprendizado de máquina com SVM \cite{Schwartz2013,aa-distortion}.
  \item Redes neurais recorrentes \cite{Bagnall2016}.
  \item Redes neurais de convolução \cite{Shrestha2017,Sari2016}.
  modelos de compressão \cite{Halvani2018}
\end{itemize}

\end{frame}
