\section{Introdução}

\begin{frame}{Introdução}
	\begin{alertblock}{Tema}
		\begin{itemize}\itemsep9pt
		\item A atribuição autoral de textos digitais (AA) (do inglês, {\it Authorship Attribution})  visa identificar quem é o autor de um determinado texto a partir de um conjunto de autores possíveis \cite{Potthast2017}.
		
		\item A premissa principal da AA é que o autor deixa rastros de seu estilo, sendo que esses rastos podem ser a preferência por certas palavras, o tamanho do vocabulário, a utilização de pontuação e a repetição de certos elementos gramaticais.
		
		\item Do ponto de vista de aprendizado de máquina, a AA pode ser vista como um problema de classificação multi-classes \cite{Stamatatos2009}.
		
		\item A quantificação do estilo de escrita, ou estilometria, compreende um vasto conjunto de medidas e técnicas que buscam extrair uma ``biometria'' textual \cite{Neal2017}.
		\end{itemize}
	\end{alertblock}

\end{frame}

\begin{frame}{Aplicações da Atribuição Autoral}

\begin{columns}
	\begin{column}{0.65\textwidth}
		\begin{block}{Aplicações da AA}
			Sua aplicação pode ajudar: 
			\begin{itemize}\itemsep9pt
				\item  em casos de escândalos de corrupção, como no caso Enron
				 \cite{corpusEnron,Chen2011}.
				\item na identificação de abusos na utilização da internet \cite{Gillam2012quite}.
				\item na detecção de notícias falsas \cite{Peng2016}.
				\item na detecção de casos onde uma pessoa tenta se passar por outra \cite{Koppel2018_pseud}.
				\item na atribuição autoral de código-fonte \cite{Alsulami2017}
				\item na detecção de pseudônimos \cite{Juola15}
				
			\end{itemize}
		\end{block}
	\end{column}
	\begin{column}{0.3\textwidth}
		\begin{block}{Áreas de interesse}
			\begin{itemize}
				\item Humanidades Digitais
				\item Análise Forense
				\item Linguística computacional
			\end{itemize}
		\end{block}
	\end{column}
\end{columns}
\end{frame}

\begin{frame}{Métodos para atribuição autoral}

Os métodos computacionais para atribuição autoral utilizam: 
\begin{itemize}
	\item Análise estatística multivariada \cite{Savoy2016,AA_delta2017}.
	\item Métodos baseados em vizinho mais próximo \cite{Kocher2017Verificacao,Koppel2018_pseud,Varela2016}.
	\item Modelos de compressão \cite{Halvani2018}.
	\item Aprendizado de máquina com SVM \cite{Schwartz2013,aa-distortion}.
	\item Redes neurais recorrentes \cite{Bagnall2016}.
	\item Redes neurais de convolução \cite{Shrestha2017,Sari2016}.
\end{itemize}

\end{frame}

\begin{frame}{Projeto - Lacunas e motivação}
	\begin{tcolorbox}[colback=blue!5!white,colframe=blue!50!black,valign=center,title=Lacunas gerais:]\selectFont
		\begin{itemize}
			\item A AA é um problema de pesquisa não totalmente resolvido \cite{Potthast2017}.
			\item É o tema da série de competições PAN-CLEF \cite{aa-overview-2018}.
			\item Estudos desta área exploram técnicas independentes de idioma e de domínio, subutilizando  recursos linguístico-computacionais, e seus avanços.
		\end{itemize}
	\end{tcolorbox}

	\begin{tcolorbox}[colback=blue!5!white,colframe=blue!50!black,valign=center,title=\citeonline{custodioParaboni2018} obteve o melhor desempenho global na PAN-CLEF2018 mas  deixa as seguintes lacunas:
		]\selectFont
		\begin{itemize}
			\item não tirou proveito de conhecimentos dependentes de idioma como (POS).
			\item e modelos de representação distribuída ({\it word embeddings}),
			\item foi restrito ao domínio {\it Fanfic},
			\item e não considerou dados em português brasileiro.
		\end{itemize}
	\end{tcolorbox}	
	

\end{frame}

\begin{frame} {Projeto - Hipóteses}

	Este trabalho considera as seguintes hipóteses:
	
	\begin{block}{H1:} O uso de modelos independentes de idioma do tipo de distorção textual permite filtrar aspectos específicos do texto, e a combinação de diversos tipos de distorção pode aumentar o desempenho de sistemas de AA.
	\end{block}
	
	\begin{block}{H2:} O uso de modelos dependentes de idioma do tipo {\it part-of-speech} extraídos por anotadores baseados em aprendizado profundo pode aumentar o desempenho de sistemas de AA.
	\end{block}
	
	
	\begin{block}{H3:} O uso de modelos dependentes de idioma do tipo representação distribuída ({\it embeddings}) pode aumentar o desempenho de sistemas de AA.
	\end{block}

\end{frame}

	\begin{frame}{Projeto - Objetivo}
	\begin{block}{Objetivo Geral}
		O objetivo geral deste trabalho é enriquecer modelos de atribuição autoral de texto digitais com conjunto fechado de autores utilizando conhecimentos dependentes e independentes de idioma combinados com técnicas de aprendizados de máquina, de modo a obter resultados superiores ao estabelecido em trabalhos anteriores.
	\end{block}
\end{frame}